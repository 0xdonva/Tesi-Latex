\chapter{Conclusioni}
\section{Principali conclusioni}
Il lavoro descritto in questa tesi ha condotto con successo all'implementazione di un honeypot e alla gestione di un IDPS, rappresentando un passo significativo verso la sicurezza informatica aziendale. L'adozione di un progetto open-source predefinito ha ottimizzato il processo, consentendo di concentrare le risorse sull'ampliamento e il miglioramento dei servizi offerti. Sebbene questo abbia semplificato la fase di installazione, l'integrazione con il SIEM ha richiesto uno sforzo aggiuntivo, evidenziando l'importanza di una comprensione approfondita del contesto e dei requisiti aziendali.\\
Il percorso seguito, partendo dalla fase di studio e definizione degli obiettivi, attraverso la progettazione e l'implementazione del progetto, ha dimostrato di essere efficace nel raggiungimento degli obiettivi prefissati. L'honeypot sviluppato è stato in grado di alleviare il traffico dannoso proveniente dagli attaccanti, mentre le competenze acquisite nell'uso di Darktrace hanno arricchito le capacità difensive aziendali.\\
Guardando al futuro, il prossimo passo sarà l'offerta di questo servizio ai clienti. A tal fine, è stata predisposta una documentazione esaustiva per facilitare l'implementazione e l'utilizzo della piattaforma, garantendo che i vantaggi e le potenzialità dell'honeypot e dell'IDPS possano essere sfruttati appieno. La continua ricerca di miglioramenti e l'adattamento alle esigenze emergenti del panorama della sicurezza informatica rimarranno al centro delle future attività.
\section{Riflessioni personali sull'esperienza del tirocinio}
Durante il tirocinio, si è avuta l'opportunità di mettere in pratica la passione per la sicurezza informatica, un ambito che ha sempre affascinato. Questo percorso formativo non è stato solo un'occasione per acquisire esperienza sul campo, ma anche per applicare e consolidare le conoscenze acquisite durante il percorso universitario. In particolare, i corsi di "Sistemi operativi", "Cyber-physical security" e "Protocolli e architetture di rete" si sono rivelati fondamentali, fornendo le competenze necessarie per affrontare sfide complesse nel contesto della sicurezza informatica. Questo tirocinio ha rappresentato un prezioso ponte tra teoria e pratica, consentendo di approfondire argomenti chiave e di affinare le abilità in un ambiente professionale.