\chapter{Conclusioni}
Il lavoro illustrato in questa tesi ha segnato un importante traguardo nell'ambito della sicurezza informatica aziendale, attraverso l'efficace implementazione di un honeypot e la gestione di un IDPS. L'adozione di un progetto open-source predefinito ha agevolato il percorso, permettendo di dedicare maggiori risorse all'espansione e al potenziamento dei servizi offerti. Sebbene ciò abbia semplificato la fase iniziale di installazione, l'integrazione con il SIEM ha richiesto un impegno sia di tempo che di studio aggiuntivo, evidenziando l'importanza di una profonda comprensione del contesto aziendale e dei suoi requisiti specifici.\\
Il percorso seguito, partendo dalla fase preliminare di studio e definizione degli obiettivi, attraverso la fase di progettazione e l'effettiva implementazione del progetto, ha dimostrato di essere un metodo efficace nel conseguimento degli obiettivi prefissati. L'honeypot sviluppato è risultato efficace nel mitigare il traffico dannoso proveniente dagli attaccanti, mentre le competenze acquisite nell'utilizzo di Darktrace hanno notevolmente arricchito le capacità difensive aziendali.\\
Per quanto riguarda i futuri sviluppi, il prossimo passo potrebbe essere quello di valutare l'offerta del servizio ai clienti, qualora vengano riscontrati benefici a lungo termine, da affiancare al monitoraggio già fornito dal SOC tramite Darktrace. In tal senso, potrebbe essere considerata l'installazione delle distribuzioni sensore di T-pot all'interno delle subnet utilizzate dai clienti. Queste distribuzioni sarebbero in grado di rilevare e bloccare qualsiasi attività sospetta tramite il SIEM, offrendo così un livello aggiuntivo di protezione.\\
Al fine di agevolare l'adozione di questa soluzione da parte dei clienti, è stata redatta una documentazione esaustiva, finalizzata a semplificare l'implementazione e l'utilizzo della piattaforma, garantendo che i vantaggi e le potenzialità dell'honeypot e dell'IDPS possano essere pienamente sfruttati. La ricerca continua di miglioramenti e l'adattamento alle esigenze emergenti del panorama della sicurezza informatica rimarranno al centro delle future attività, con l'obiettivo primario di garantire una protezione sempre più efficace contro le minacce in continua evoluzione.