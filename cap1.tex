\chapter{Introduzione}
Nell'ambito della sicurezza informatica, una delle sfide principali è l'identificazione tempestiva e l'analisi approfondita delle minacce. Le organizzazioni, pubbliche e private, devono affrontare l'inevitabile complessità delle reti, la diversità delle minacce sia interne che esterne alla rete privata. Ciò in quanto la mancanza di strumenti efficaci per il rilevamento delle intrusioni e la gestione delle minacce può portare a vulnerabilità e rischi per la sicurezza.\\
IBM stima che nel 2023 il costo medio di un data breach è di 4,5 milioni di dollari americani. In questo contesto, dove le minacce cibernetiche sono in costante evoluzione, diventa cruciale sviluppare e implementare metodologie avanzate per proteggere i dati, gli utenti, i dispositivi e le reti dalle intrusioni, utilizzando tecniche proattive per rilevare, prevenire e rispondere agli attacchi informatici.\\
Tra esse è possibile annoverare l’utilizzo di un honeypot, come T-Pot, oppure di un IDPS, come Darktrace. Questi sistemi, approfonditi nel presente lavoro, sono stati utilizzati durante l’esperienza di tirocinio al fine di mitigare minacce interne ed esterne alla rete interna.\\
Rispetto alle soluzioni tradizionali, che si basano principalmente su firewall per il perimetro esterno e policy di routing e VLAN per il perimetro interno, l'implementazione di un honeypot e lo studio delle possibili minacce rilevabili tramite IDPS durante il tirocinio offrono una maggiore visibilità delle minacce in tempo reale. Ciò consente una risposta più rapida e mirata agli attacchi, grazie all'integrazione di IDPS e honeypot, aumenta la possibilità di prevenire incidenti tramite la deviazione di attacchi nel caso sia al di fuori del perimetro o tramite il monitoraggio della rete nel caso sia all'interno.\\
Attraverso la sperimentazione dell'honeypot verranno descritte le principali minacce e vettori di attacco nel mondo della sicurezza informatica, inoltre grazie a Darktrace verranno esaminati diversi potenziali scenari di attacco e problemi giornalieri specificando le procedure per mitigare i rischi.\\
Il lavoro svolto si concentra sulla ricerca di soluzioni pratiche e efficaci per affrontare le sfide sempre crescenti nel campo della sicurezza informatica, fornendo un contributo significativo alla prevenzione di possibili attacchi e alla diminuzione del traffico malevolo su sistemi in uso.
\begin{quote}
	{\small Nel capitolo $2$ viene esaminata la teoria relativa agli honeypot e agli IDS, approfondendo le definizioni e le tipologie esistenti. Si analizzano poi in dettaglio le piattaforme utilizzate per implementare tali concetti.\\
	Nel capitolo $3$ vengono esplorati gli utilizzi pratici delle suddette piattaforme, includendo configurazioni, dati raccolti e esempi di attacchi rilevati.\\
	Infine, nel capitolo $4$ viene presentata la conclusione, riassumendo le principali scoperte e riflessioni emerse durante lo studio e l'implementazione delle soluzioni discusse.}
\end{quote}