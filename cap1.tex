\chapter{Introduzione}
Nell'ambito della sicurezza informatica, una delle sfide principali è l'identificazione tempestiva e l'analisi approfondita delle minacce. Le organizzazioni, pubbliche e private, devono affrontare l'inevitabile complessità delle reti, la diversità delle minacce sia interne che esterne alla rete privata. Ciò in quanto la mancanza di strumenti efficaci per il rilevamento delle intrusioni e la gestione delle minacce può portare a vulnerabilità e rischi per la sicurezza.\\
IBM stima che nel 2023 il costo medio di un data breach è di 4,5 milioni di dollari americani. In questo contesto, dove le minacce cibernetiche sono in costante evoluzione, diventa cruciale sviluppare e implementare metodologie avanzate per proteggere i dati, gli utenti, i dispositivi e le reti dalle intrusioni, utilizzando tecniche proattive per rilevare, prevenire e rispondere agli attacchi informatici.\\
Tra esse è possibile annoverare l’utilizzo di un honeypot, come T-Pot, oppure di un IDPS, come Darktrace. Questi sistemi, approfonditi nel presente lavoro, sono state utilizzate durante l’esperienza di tirocinio al fine di mitigare minacce interne ed esterne alla rete interna.\\
Rispetto alle soluzioni tradizionali, l’approccio proposto durante il tirocinio offre una maggiore visibilità delle minacce in tempo reale, consentendo una risposta più rapida e mirata agli attacchi: integrando IDPS e honeypot, si ottiene una difesa a strati più robusta e una capacità avanzata di rilevamento delle minacce sia note che sconosciute.\\
Attraverso la sperimentazione dell'honeypot verranno descritte le principali minacce nel mondo della sicurezza informatica, inoltre verranno esaminati diversi scenari di attacco specificando le procedure per mitigare problemi al cliente.\\
Il lavoro svolto si concentra sulla ricerca di soluzioni pratiche e efficaci per affrontare le sfide sempre crescenti nel campo della sicurezza informatica, fornendo un contributo significativo alla protezione delle reti e dei sistemi informatici dalle minacce cibernetiche.